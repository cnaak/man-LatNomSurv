%------------------------------------------------------------------------------------------------------------------------------%
%                                                           Packages                                                           %
%------------------------------------------------------------------------------------------------------------------------------%
\usepackage{booktabs}
\usepackage[english]{babel}
\usepackage[squaren,cdot]{SIunits}
\usepackage{amsmath}
\usepackage{amssymb}
\usepackage{amsthm}
\usepackage{textcomp}
\usepackage{pslatex}
\usepackage[lining]{ebgaramond}
\usepackage{indentfirst}
\usepackage{xspace}
%-------------------------------------------------------------------------------------------------------------------------------
\usepackage[hyperindex,breaklinks]{hyperref} % Required for hyperlinks
%------------------------------------------------------------------------------------------------------------------------------%
%                                                           Commands                                                           %
%------------------------------------------------------------------------------------------------------------------------------%
\hypersetup{%
    hidelinks,
    colorlinks,
    breaklinks=true,
    urlcolor=color3,
    citecolor=color1,
    linkcolor=color1,
    bookmarksopen=false,
    pdftitle={Title},
    pdfauthor={Author}
}
%-------------------------------------------------------------------------------------------------------------------------------
\setlength{\abovecaptionskip}{4pt}
\setlength{\columnsep}{5.5mm}
\setlength{\columnseprule}{0.2pt}
\setlength{\fboxrule}{0.4pt} % Width of the border around the abstract
%-------------------------------------------------------------------------------------------------------------------------------
\definecolor{color1}{RGB}{0,0,90} % Color of the article title and sections
\definecolor{color2}{RGB}{0,20,20} % Color of the boxes behind the abstract and headings
\definecolor{color3}{RGB}{0,0,192} % Color of the article title and sections
%-------------------------------------------------------------------------------------------------------------------------------
\newtheorem{theorem}{Theorem}
\newtheorem{definition}{Definition}
%-------------------------------------------------------------------------------------------------------------------------------
\newcommand{\XXX}[1]{\relax}
%------------------------------------------------------------------------------------------------------------------------------%
%                                                           Metadata                                                           %
%------------------------------------------------------------------------------------------------------------------------------%
\makeatletter
\immediate\write18{datelog > \jobname.info}
\makeatother
%-------------------------------------------------------------------------------------------------------------------------------
% Journal information
\JournalInfo{engrXiv}
\Archive{Compiled on \input{\jobname.info} -- Version 0
}
% Article title
\PaperTitle{Lattice Nomenclature Survey from LGA to Modern LBM}
\Authors{%
    C.~Naaktgeboren\textsuperscript{1$\star$},
    F.~N.~de Andrade\textsuperscript{2}
}
\affiliation{%
    \textsuperscript{1}%
    \textit{%
        Adjunct Professor.
        Universidade Tecnológica Federal do Paraná -- UTFPR, Câmpus Guarapuava.
        Grupo de Pesquisa em Ciências Térmicas.
}}
\affiliation{%
    \textsuperscript{2}%
    \textit{%
        Universidade Tecnológica Federal do Paraná -- UTFPR, Câmpus Guarapuava.
        Grupo de Pesquisa em Ciências Térmicas.
}}
\affiliation{%
    \textsuperscript{$\star$}%
    \textbf{Corresponding  author}: NaaktgeborenC$\cdot$PhD\textcircled{a}gmail$\cdot$com
}
\Keywords{%
    lattice-Boltzmann stencils ---
    lattice-Boltzmann models ---
    lattice nomenclature systems ---
    high-order lattices.
}
\newcommand{\keywordname}{Keywords}
\Highlights{%
    Surveys lattice nomenclature systems in the hystory of LBM: from the LGA to current times ---
    Discusses lattice categories, competing standards, naming ambiguity, model parameter scope.
}
\newcommand{\highlightname}{Highlights}
%-------------------------------------------------------------------------------------------------------------------------------
\Abstract{%
    Lattice configuration is a core parameter in Lattice-Boltzmann  (LB)  methods,  both  from  theoretical  and  implementation
    standpoints. As LB methods have progressed over the past decades, a variety of lattice configurations have been proposed and
    refered to according to a plurality  of  lattice  nomenclature  systems  that  usually  include  the  Euclidean  \emph{space
    dimensionality}, the lattice \emph{velocity count} and, in fewer instances, the \emph{discretization order} in their format.
    This work surveys lattice nomenclature systems, or lattice naming schemes, along the history of LB  methods,  starting  from
    their Lattice Gas Automata (LGA) predecessor method, up to the present time. Findings include multiple  lattice  categories,
    competing naming standards, ambiguous names particularly in higher-order models, naming systems of varying  model  parameter
    scopes, and lack of unambiguous naming schemes even for space-filling, Bravais lattice types.
}
%-------------------------------------------------------------------------------------------------------------------------------
