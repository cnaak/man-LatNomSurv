%------------------------------------------------------------------------------------------------------------------------------%
%                                                          Title Page                                                          %
%------------------------------------------------------------------------------------------------------------------------------%

\thispagestyle{empty} % Removes page numbering from the first page
\flushbottom % Makes all text pages the same height
\maketitle % Print the title and abstract box


%------------------------------------------------------------------------------------------------------------------------------%
%                                                           License                                                            %
%------------------------------------------------------------------------------------------------------------------------------%

\section*{License}

    \scriptsize\noindent%
    \begin{minipage}{\columnwidth}
        \centering\tt
        \includegraphics[height=6.0mm]{cc/by.pdf}\\[0.5\smallskipamount]
        {\scriptsize\url{https://creativecommons.org/licenses/by/4.0/}}
    \end{minipage}
    \normalsize


%------------------------------------------------------------------------------------------------------------------------------%
%                                                      Table Of Contents                                                       %
%------------------------------------------------------------------------------------------------------------------------------%

\tableofcontents


%------------------------------------------------------------------------------------------------------------------------------%
%                                                         Introduction                                                         %
%------------------------------------------------------------------------------------------------------------------------------%

\section{Introduction}

    Historically,   the   Lattice-Boltzmann   (LB)    method    had    its    origins    in    the    frame    of    Lattice-Gas
    Automata~\cite{1988-McNamaraGR+ZanettiG-PhysRevLett},     and     has     been     intensely     developed     since     its
    inception~\cite{1992-BenziR+VergassolaM-PhysRep,       1998-ChenS+DoolenGD-AnnuRevFluidMech,        2011-MohamadAA-Springer,
    2018-KrugerT+ViggenEM-Springer}. One important conceptual and implementational parameter  of  LB  methods  is  the  employed
    \emph{lattice    stencil}---understood    as    the    lattice    geometry,    velocity    set,    weights,    and     scale
    parameters~\cite{2013-HegeleJr+PhilippiPC-JSciComput,                                2013-MattilaKK+PhilippiPC-IntJModPhysC,
    2014-MattilaKK+PhilippiPC-SciWorldJ}, although some authors may include  in  the  stencil  designation  additional  modeling
    elements, such as the relaxation time scheme~\cite{2017-LiL+KlausnerJF-IntJHeatMassTran}.

    Both LB and LGA methods can be implemented on a variety of lattices, and historically many such lattices (along  with  their
    corresponding names or naming systems) have been developed. This work presents a LB literature  survey  focused  on  lattice
    naming schemes, or model nomenclature systems, from its Lattice-Gas Automata (LGA)  predecessor  method  until  the  present
    time, in a somewhat chronological timeline.


%------------------------------------------------------------------------------------------------------------------------------%
%                                                 Lattice Nomenclature Survey                                                  %
%------------------------------------------------------------------------------------------------------------------------------%

\section{Lattice Nomenclature Survey}

    %---------------------------------------------------------------------------------------------------------------------------
    \subsection{Lattice-Gas Automata Lattice Designations}

    Some   LGA   lattices   were   named   with   \emph{acronyms}   after   its   first   proposers,   such   as    the    `HPP'
    one~\cite{1986-FrischU+PomeauY-PhysRevLett},  after  Hardy,  de  Pazzis,   and   Pomeau~\cite{1973-HardyJ+PazzisO-JMathPhys,
    1976-HardyJ+PomeauY-PhysRevA,  1987-SucciS-JPhysAMathGen},  or   geometry-based   \emph{acronyms},   such   as   the   `HLG'
    one~\cite{1986-FrischU+PomeauY-PhysRevLett}, which stands for `hexagonal lattice gas,' later on referred  to  as  the  `FHP'
    one~\cite{1987-FrischU+RivetJP-ComplexSyst,  1987-SucciS-JPhysAMathGen},  after  Frisch,  Hasslacher,  and  Pomeau.  Another
    geometry-based  lattice  of  the  time  is  the  `FCHC'  one~\cite{1987-FrischU+RivetJP-ComplexSyst},   which   stands   for
    `face-centered-hypercubic' model, due to d'Humières, Lallemand, and Frisch.

    Later on designations such as `FHP + 3 rest particles' and `FCHC + 3 rest particles' also appeared~\cite{1991-BoonJP-PhysD},
    as   well   as   suffixes   such   as   `-I',   `-III',   and    `-IV'    after    `FHP',    for    alternative    collision
    rules~\cite{1991-AppertC+ZaleskiS-PhysD, 1991-BoonJP-PhysD, 1991-ChenS+RoseH-PhysD}.

    %---------------------------------------------------------------------------------------------------------------------------
    \subsection{Early Lattice-Boltzmann Years}

    \vspace{2.0mm}\noindent\textbf{Inception Period:}\vspace{1.0mm}

    LB   methods   adhered    to    LGA    lattice    nomenclature    in    its    inception    period,    as    witnessed    by
    reference~\cite{1988-McNamaraGR+ZanettiG-PhysRevLett}          in          1988          and          by          subsequent
    references~\cite{1989-HigueraFJ+JimenezJ-EurophysLett,         1989-HigueraFJ+SucciS-EurophysLett}         in          1989,
    by~\cite{1990-BenziR+VergassolaM-EurophysLett,   1990-BenziR+VergassolaM-NuclPhysB,    1990-CancelliereA+SucciS-PhysFluidsA,
    1990-VergassolaM+SucciS-EurophysLett}   in   1990,   and   by~\cite{1991-CornubertR+LevermoreD-PhysD,    1991-ErnstMH-PhysD,
    1991-FrischU-PhysD, 1991-GunstensenAK+ZanettiG-PhysRevA, 1991-SucciS+BenziR-PhysRevA} in 1991, to cite a few.

    \vspace{2.0mm}\noindent\textbf{Early 1990's:}\vspace{1.0mm}

    It seems that Qian~\cite{1990-QianYH-Paris} (apud~\cite[p.~235]{1993-QianYH-JSciComput}) was the one to introduce, in  1990,
    the `DdQb' lattice naming scheme for LB methods---in which $d$ is the lattice \emph{Euclidean dimensionality} and $b$ is the
    lattice  \emph{velocity   count},   as   in   D1Q3,   D2Q9,   and   D3Q15,   etc.~\cite{1992-QianYH+LallemandP-EurophysLett,
    1993-QianYH+OrszagSA-EuroPhysLett}---that seems to be the most prevalent lattice naming system  to  date,  although  notable
    exceptions appear long after the paper~\cite{1991-QianYH+LallemandP-AdvKinTheoContMech} came out in 1991.



%------------------------------------------------------------------------------------------------------------------------------%
%                                                          Discussion                                                          %
%------------------------------------------------------------------------------------------------------------------------------%

\section{Discussion}


%------------------------------------------------------------------------------------------------------------------------------%
%                                                         Conclusions                                                          %
%------------------------------------------------------------------------------------------------------------------------------%

\section{Conclusions}


%-------------------------------------------------------------------------------------------------------------------------------
