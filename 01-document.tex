%------------------------------------------------------------------------------------------------------------------------------%
%                                                          Title Page                                                          %
%------------------------------------------------------------------------------------------------------------------------------%

\thispagestyle{empty} % Removes page numbering from the first page
\flushbottom % Makes all text pages the same height
\maketitle % Print the title and abstract box


%------------------------------------------------------------------------------------------------------------------------------%
%                                                           License                                                            %
%------------------------------------------------------------------------------------------------------------------------------%

\section*{License}

    \scriptsize\noindent%
    \begin{minipage}{\columnwidth}
        \centering\tt
        \includegraphics[height=6.0mm]{cc/by.pdf}\\[0.5\smallskipamount]
        {\scriptsize\url{https://creativecommons.org/licenses/by/4.0/}}
    \end{minipage}
    \normalsize


%------------------------------------------------------------------------------------------------------------------------------%
%                                                      Table Of Contents                                                       %
%------------------------------------------------------------------------------------------------------------------------------%

\tableofcontents


%------------------------------------------------------------------------------------------------------------------------------%
%                                                         Introduction                                                         %
%------------------------------------------------------------------------------------------------------------------------------%

\section{Introduction}

    Historically,   the   lattice-Boltzmann   (LB)    method    had    its    origins    in    the    frame    of    Lattice-Gas
    Automata~\cite{1988-McNamaraGR+ZanettiG-PhysRevLett},     and     has     been     intensely     developed     since     its
    inception~\cite{1992-BenziR+VergassolaM-PhysRep,       1998-ChenS+DoolenGD-AnnuRevFluidMech,        2011-MohamadAA-Springer,
    2018-KrugerT+ViggenEM-Springer}. One important conceptual and  implementation  parameter  of  LB  methods  is  the  employed
    \emph{lattice    stencil}---understood    as    the    lattice    geometry,    velocity    set,    weights,    and     scale
    parameters~\cite{2013-HegeleJr+PhilippiPC-JSciComput,                                2013-MattilaKK+PhilippiPC-IntJModPhysC,
    2014-MattilaKK+PhilippiPC-SciWorldJ}, although some authors may include  in  the  stencil  designation  additional  modeling
    elements, such as the relaxation time scheme~\cite{2017-LiL+KlausnerJF-IntJHeatMassTran}.

    Both LB and LGA methods can be implemented on a variety of lattices, and historically many such lattices (along  with  their
    corresponding names or naming systems) have been developed. This work presents a LB literature  survey  focused  on  lattice
    naming schemes, or model nomenclature systems, from its Lattice-Gas Automata (LGA)  predecessor  method  until  the  present
    time, in a somewhat chronological timeline.


%------------------------------------------------------------------------------------------------------------------------------%
%                                                 Lattice Nomenclature Survey                                                  %
%------------------------------------------------------------------------------------------------------------------------------%

\section{Lattice Nomenclature Survey}

    %---------------------------------------------------------------------------------------------------------------------------
    \subsection{Lattice-Gas Automata Lattice Designations}

    Some  LGA  lattices   were   named   with   \emph{ac\-ro\-nyms}   after   its   first   proposers,   such   as   the   `HPP'
    one~\cite{1986-FrischU+PomeauY-PhysRevLett},  after  Hardy,  de  Pazzis,   and   Pomeau~\cite{1973-HardyJ+PazzisO-JMathPhys,
    1976-HardyJ+PomeauY-PhysRevA,  1987-SucciS-JPhysAMathGen},  or  geometry-based  \emph{ac\-ro\-nyms},  such  as   the   `HLG'
    one~\cite{1986-FrischU+PomeauY-PhysRevLett}, which stands for `hexagonal lattice gas,' later on referred  to  as  the  `FHP'
    one~\cite{1987-FrischU+RivetJP-ComplexSyst,  1987-SucciS-JPhysAMathGen},  after  Frisch,  Hasslacher,  and  Pomeau.  Another
    geometry-based  lattice  of  the  time  is  the  `FCHC'  one~\cite{1987-FrischU+RivetJP-ComplexSyst},   which   stands   for
    `face-centered-hypercubic' model, due to d'Hu\-mi\-è\-res, Lallemand, and Frisch.

    Later on designations such as `FHP + 3 rest particles' and `FCHC + 3 rest particles' also appeared~\cite{1991-BoonJP-PhysD},
    as   well   as   suffixes   such   as   `-I',   `-III',   and    `-IV'    after    `FHP',    for    alternative    collision
    rules~\cite{1991-AppertC+ZaleskiS-PhysD, 1991-BoonJP-PhysD, 1991-ChenS+RoseH-PhysD}.


    %---------------------------------------------------------------------------------------------------------------------------
    \subsection{Early Lattice-Boltzmann Years}


    \vspace{2.0mm}\noindent\textbf{Inception Period:}\vspace{1.0mm}

    LB   methods   adhered    to    LGA    lattice    nomenclature    in    its    inception    period,    as    witnessed    by
    reference~\cite{1988-McNamaraGR+ZanettiG-PhysRevLett}          in          1988          and          by          subsequent
    references~\cite{1989-HigueraFJ+JimenezJ-EuroPhysLett,         1989-HigueraFJ+SucciS-EuroPhysLett}         in          1989,
    by~\cite{1990-BenziR+VergassolaM-EuroPhysLett,   1990-BenziR+VergassolaM-NuclPhysB,    1990-CancelliereA+SucciS-PhysFluidsA,
    1990-VergassolaM+SucciS-EuroPhysLett}   in   1990,   and   by~\cite{1991-CornubertR+LevermoreD-PhysD,    1991-ErnstMH-PhysD,
    1991-FrischU-PhysD, 1991-GunstensenAK+ZanettiG-PhysRevA, 1991-SucciS+BenziR-PhysRevA} in 1991, to cite a few.


    \vspace{2.0mm}\noindent\textbf{Early 1990's:}\vspace{1.0mm}

    It seems that Qian~\cite{1990-QianYH-Paris} (apud~\cite[p.~235]{1993-QianYH-JSciComput}) was the one to introduce, in  1990,
    the `DdQb' lattice naming scheme for LB methods---in which $d$ is the lattice \emph{Euclidean dimensionality} and $b$ is the
    lattice  \emph{velocity   count},   as   in   D1Q3,   D2Q9,   and   D3Q15,   etc.~\cite{1992-QianYH+LallemandP-EuroPhysLett,
    1993-QianYH+OrszagSA-EuroPhysLett}---that seems to be the most prevalent lattice naming system  to  date,  although  notable
    exceptions appear long after the paper~\cite{1991-QianYH+LallemandP-AdvKinTheoContMech} came out in 1991.

    As far as increasing lattice velocity counts go, the relationship between mesoscopic lattice \emph{symmetry}  and  resulting
    macroscopic   description   \emph{isotropy}   has   been   established    from    early    in    the    history    of    LGA
    methods~\cite{1973-HardyJ+PazzisO-JMathPhys, 1976-HardyJ+PomeauY-PhysRevA},  in  two~\cite{1986-FrischU+PomeauY-PhysRevLett}
    and in three Euclidean dimensions, the latter requiring the lattice to include links beyond nearest neighbors~\cite[pp.~473,
    490]{1986-WolframS-JStatPhys}, hence particle velocities with unequal magnitudes.

    Moreover, Koelman~\cite{1991-KoelmanJMVA-EuroPhysLett} had proposed matching discrete velocity moments up to a certain order
    $n$ with the $d$-dimensional continuous Boltzmann distribution, since only those  moments  influence  the  macroscopic  flow
    behavior; such procedure would yield values for lattice velocity \emph{weights} $W_{\alpha}$.  The  proposed  criteria  were
    deemed more stringent than previously well-known symmetry  and  isotropy  requirements  from~\cite{1986-WolframS-JStatPhys},
    since it not only led to an isotropic macroscopic description, but also ensure  pressure  term  independence  from  velocity
    terms of the Navier-Stokes description. Furthermore, a skewed rectangular 9-speed lattice with independent $a$ and $b$  axis
    lengths was proposed\footnote{That lattice was named `face-centered rectangular' by  the  author.},  whose  weights  exactly
    recover those of the well-known D2Q9 lattice for $a = b$, over which the argument that valid weights `[...] \emph{can always
    be found by choosing a large enough set of} (lattice) \emph{velocity vectors\/} [...]'~\cite{1991-KoelmanJMVA-EuroPhysLett}.

    One    driving    application     for     increased     velocity     count     lattices     is     thermal     flows.     On
    reference~\cite{1993-AlexanderFJ+SterlingJD-PhysRevE} an \emph{unnamed}  2D,  hexagonal  (triangular),  13-velocity  lattice
    having velocity magnitudes of 0,  1,  and  2  lattice  units~\cite{1998-ChenS+DoolenGD-AnnuRevFluidMech}  was  employed  for
    adiabatic sound propagation and heat transfer Couette flow, whose results were shown to be in agreement  with  corresponding
    analytical solutions.

    Some `nDmV' lattices, with $n$ being the Euclidean space dimension and $m$ the lattice velocity count, namely, 1D5V,  2D16V,
    and 3D40V, were introduced in~\cite{1994-ChenY+AkiyamaM-PhysRevE} for  shock  wave  front  structure  and  shear  wave  flow
    application cases. The 2D16V lattice, for instance, was said to be  comprised  of  four  \emph{sub-lattices}---a  term  that
    appeared in subsequent references---with each sublattice having 4 discrete velocities of same magnitude and forming adjacent
    right angles, which led to possibly \emph{multiple sublattices per lattice energy level} $\epsilon \equiv 2e  =  c^2$,  with
    $c$ being the microscopic (lattice) velocity magnitude, and $e$ the corresponding specific kinetic energy, as was  the  case
    with the $\epsilon = 1^2 + 2^2 = 5$ energy level of a square lattice, represented by the 8 discrete velocities obtained from
    permutations of ($\pm\{1,2\}$, $\pm\{2,1\}$) in lattice units, which were grouped in two distinct sublattices.  This  is  in
    contrast to later works in which energy levels are treated as single groups.


    \vspace{2.0mm}\noindent\textbf{Late 1990's to mid-2000's:}\vspace{1.0mm}

    Most likely borrowing from mesh-based continuous mechanics numerical methods, a study~\cite{1996-HeX+DemboM-JComputPhys} has
    proposed a LB algorithm for non-uniform mesh grids, by decoupling spatial and  momentum  space  discretizations  in  the  LB
    scheme. The underlying momentum space discretization was the well-known D2Q9 lattice, referred to in the study as `9-bit BGK
    model  in  2D  space'  and  other  semantically  equivalent  sentences,  in  which   BGK   stands   for   kinetic   theory's
    Bhatnagar-Gross-Krook  collision  model  for  the   continuous   Boltzmann   equation~\cite{1954-BhatnagarPL+KrookM-PhysRev,
    2003-LiboffRL-bookSpringer, 2011-HarrisS-Dover}.

    Nine years after the debut of LB methods, a study~\cite{1997-HeX+LuoLS-PhysRevE} showed that they could be directly  derived
    from    the    continuous    Boltzmann    equation    with    linearized    collision     operator     under     the     BGK
    approximation~\cite{2011-HarrisS-Dover}, while lattice stencils  from  requirements  of  matching  continuous  and  discrete
    velocity moments up to a desired order---a decisive publication, not only in making LB methods theory independent  from  its
    LGA historical predecessor, but also to pave the way towards later methods for lattice weights determination for the lattice
    velocity     set     based     on      some      discrete-to-continuous      equivalences~\cite{2006-ShanX+ChenH-JFluidMech,
    2006-PhilippiPC+SurmasR-PhysRevE}.  The  lattices  in~\cite{1997-HeX+LuoLS-PhysRevE}   were   verbosely   referred   to   as
    `$d$-dimensional $b$-bit $g$ lattice model', with $d$ being the Euclidean space dimension, $b$ the lattice  velocity  count,
    and $g$ a geometry term, such as `triangular,' etc.

    A review article by Shen and  Doolen~\cite{1998-ChenS+DoolenGD-AnnuRevFluidMech}  published  a  decade  after  McNamara  and
    Zanetti's premiere LB publication~\cite{1988-McNamaraGR+ZanettiG-PhysRevLett} and seven years after Qian's paper introducing
    the now-prevailing `DdQb' lattice naming scheme~\cite{1991-QianYH+LallemandP-AdvKinTheoContMech}, would still  refer  to  LB
    lattices either with LGA-style or verbose nomenclatures, and to overall LB schemes based on its  collision  term  treatment,
    such as `lattice BGK (LBGK)' models.

    Higher-order lattices were proposed  in~\cite{1998-PavloP+VahalaL-PhysRevLett}  for  two-  and  three-dimensional  Euclidean
    spaces. They were referred to as `octagonal grid (17-bit),' and as `3D ``octagonal'' 53-bit' models, respectively, and  were
    isotropic up to the sixth-order. Since octagons are not space-filling, plane-tiling geometries, the proposed  lattices  were
    not of the Bravais type, meaning they impose a decoupling between the spatial and the  momentum  space  discretizations,  as
    with the non-uniform mesh~\cite{1996-HeX+DemboM-JComputPhys}, and the method has to  resort  to  interpolations,  which  was
    later shown to cause spurious numerical diffusion~\cite[p.~429]{2006-ShanX+ChenH-JFluidMech}.

    Other     lattice     namings     of     the     early-     and     mid-2000's     include     verbose,     spelled      out
    ones~\cite{2001-dHumieresD+LallemandP-PhysRevE,   2005-LuXY-IntJModPhysC};   a   `$D_dQ_b$'   variant   of   Qian's   `DdQb'
    scheme~\cite{2003-NourgalievRR+JosephD-IntJMulFlow};     a     `groupI'     to     `groupIV'     regular     2D      polygon
    variant~\cite{2003-WatariM+TsutaharaM-PhysRevE, 2007-WatariM-PhysA}; a `$b$  ($d$D)'  short  designation  for  an  otherwise
    verbose one~\cite{2006-ChikatamarlaSS+KarlinIV-PhysRevLett}; an explicit lattice units velocity list, such as `$\{0, \pm  1,
    \pm 3\}$', in~\cite{2006-ChikatamarlaSS+KarlinIV-PhysRevLett}; and the `dodecahedron' and `icosahedron' ones that were shown
    to be stable for supersonic thermal flows~\cite{2006-WatariM+TsutaharaM-PhysA, 2007-WatariM-PhysA}.


    %---------------------------------------------------------------------------------------------------------------------------
    \subsection{The Year of 2006}

    The year of 2006 is seemingly a landmark for multi-velocity, higher-order LB schemes---and incidentally for  lattice  naming
    schemes---as evidenced by the appearance of three key publications, namely those of  Shan,~X.,  Yuan,~X.-F.,  and  Chen,~H.,
    \cite{2006-ShanX+ChenH-JFluidMech},   of   Philippi,~P.~C.,   Hegele,~L.~A.,   dos    Santos,~L.~O.~E.    and    Surmas,~R.,
    \cite{2006-PhilippiPC+SurmasR-PhysRevE},        and        of         Chikatamarla,~S.~S.         and         Karlin,~I.~V.,
    \cite{2006-ChikatamarlaSS+KarlinIV-PhysRevLett}.


    \vspace{2.0mm}\noindent\textbf{Shan and Coauthors:}\vspace{1.0mm}

    A  systematic   discretization   framework   for   the   Boltzmann   equation   was   proposed   by   Shan   and   coauthors
    in~\cite{2006-ShanX+ChenH-JFluidMech}.  From  kinetic  theory~\cite{2011-HarrisS-Dover,   2003-LiboffRL-bookSpringer},   the
    authors pointed out  that  successive  Chapman-Enskog  approximations  of  the  Boltzmann  equation  obtain  the  (i)~Euler,
    (ii)~Navier-Stokes, (iii)~Burnett, and (iv)~high\-er-or\-der macroscopic equations---meaning progressively  high\-er-or\-der
    moments  of  the  continuous  Boltzmann  equation  express  progressively  high\-er-or\-der  macroscopic  thermohydrodynamic
    descriptions. Moreover, the authors demonstrated that projecting the Boltzmann equation onto order-$N$  truncated  tensorial
    Hermite polynomial expansion bases~\cite{1949-GradH-CommPureApplMath}, lead  to  discrete  LB  models  of  corres\-pon\-ding
    order-$N$ moments, since resulting Hermite expansion coefficients correspond to the velocity moments up to the chosen order.

    In this discretization framework, the lattice is viewed as a Hermite expansion \emph{quadrature}, and the naming  convention
    was defined in terms of three parameters, namely, an Euclidean space dimension $D$, a quadrature velocity count $d$, and  an
    algebraic degree of precision $n$ encoded in an `$E_{D,n}^{d}$' naming scheme---an order-$N$ Hermite  expansion  requires  a
    quadrature degree $n \geqslant 2N$. Citing Qian  and  coauthors'  lattices~\cite{1992-QianYH+LallemandP-EuroPhysLett},  they
    established the following comparisons, which were off only by a scaling factor: $D2Q9 \propto E_{2,5}^{9}$,  $D3Q15  \propto
    E_{3,5}^{15}$, and $D3Q19 \propto E_{3,5}^{19}$.

    Additionally, they established that Gauss-Hermite quadratures of the Boltzmann equation yield LB models  with  \emph{minimum
    velocity count} for a given degree of precision and Euclidean spacial dimension, without, however, the ability to  predefine
    (choose) the discrete velocity abscisae,  which  apart  from  special  cases  fails  to  produce  a  space-filling,  Bravais
    lattice---recalling that for LB methods, this means lower memory requirements but decoupled spatial  and  momentum  `meshes'
    that require interpolations, thus introducing artifacts such as spurious numerical diffusion.

    In the  Appendix  of  reference~\cite{2006-ShanX+ChenH-JFluidMech},  the  authors  include  a  brief  overview  on  deriving
    quadratures on predefined Cartesian abscissae, which is  the  main  requirement  for  space-filling,  Bravais  lattices  for
    non-interpolating, exact advection LB schemes. The brief overview, however, is of scalar nature, while a tensorial treatment
    is needed for full clarity. Results for the space-filling $E_{2,7}^{17}$ and $E_{3,7}^{39}$ quadratures  were  listed  among
    the ones obtained with Gauss-Hermite quadratures.

    % Missing details include (i) how to switch from rank-0 polynomial base to the tensorial Hermite one, (ii) and from scalar
    % argument ξ to a vectorial 𝛏 one.


    \vspace{2.0mm}\noindent\textbf{Philippi and Coauthors:}\vspace{1.0mm}

    Tackling the aspects associated in deriving space-filling, Bravais  lattices  aiming  at  sufficiently  high  orders  as  to
    approach thermal  hydrodynamic  transport  problems,  Philippi  and  coauthors~\cite{2006-PhilippiPC+SurmasR-PhysRevE}  have
    proposed a new \emph{Method of Prescribed Abscissas\/}, MPA, for obtaining lattice weight values  and  scaling  factor  from
    predefined lattice arrangements.

    Departing from the continuous Boltzmann equation, the derivation of discrete velocity sets, i.e., the lattice  vectors,  and
    corresponding weights, was considered as a quadrature problem aiming at (i)~matching discrete equilibrium mass  distribution
    function with its continuous counterpart and at (ii)~warranting  even-ranked  velocity  tensor  isotropy,  which,  in  turn,
    translates into isotropic fluid transport properties.

    The Method of Prescribed Abscissas, MPA, yields \emph{implicit} equations for lattice weights and lattice  scale  factor  in
    the form of polynomial tensor products, which are generally excessively numerous, especially for  higher-order  cases.  They
    have to be selected (reduced) and converted either into a non-linear system of equations. The apparent  lack  of  literature
    guidance   in    tensor    component    equation    selection    criteria    and    solution    approach    has    motivated
    works~\cite{2016-AndradeFN-MEngUTFPR, 2019-daRosaTG-MEngUTFPR}.

    In their prescribed abscissas quadrature discussion, authors state that~\cite[p.~6]{2006-PhilippiPC+SurmasR-PhysRevE}:
    %
    \begin{quote}
        \swshape
        ``[...]~$N$th-order approximation to the [Maxwell-Boltzmann]  equilibrium  distribution  is  required  when  $N$th-order
        macroscopic equilibrium moments are to be correctly described in [lattice-Boltzmann methods,]''
    \end{quote}
    %
    \noindent which is homologous to many Shan and  coauthor's  statements  in~\cite{2006-ShanX+ChenH-JFluidMech}.  Observations
    like these, allied to the new and consistent methods of deriving higher-order LB stencils  by  Gauss-Hermite  or  Prescribed
    Abscissas quadratures, allowed for the immediate and subsequent appearances of lattices in 2D- and  in  3D-Euclidean  spaces
    with increased velocity counts, many of which requiring changes or adaptations in the naming scheme, as  the  sequence  will
    show.

    Immediate examples~\cite{2006-PhilippiPC+SurmasR-PhysRevE} include (i)~\emph{two} forms of bi-dimensional, 17-velocity ones,
    named D2Q17 and D2V17 for distinction; (ii)~a D2Q21 one; as well as (iii)~\emph{two}  forms  of  bidimensional,  25-velocity
    ones, named D2V25(W1) and D2V25(W6), containing the energy levels $\epsilon \in \{0,$ $2,$ $4,$ $8,$ $9,$ $16,$ $18\}$,  and
    $\epsilon \in \{0,$ $1,$ $2,$ $4,$ $8,$ $9,$ $16\}$, respectively---thus without the energy levels (hence, weights)  labeled
    `1' and `6' for `(W1)' and `(W6)', respectively---and (iv)~a  fourth-order  ($N=4$)  Hermite  D2V37  lattice,  suitable  for
    thermal flow LB simulations.


    \vspace{2.0mm}\noindent\textbf{Chikatamarla and Karlin:}\vspace{1.0mm}

    Seeking  to  systematically   derive   \emph{stable}   and   \emph{Galilean   invariant}   LB   models,   Chikatamarla   and
    Karlin~\cite{2006-ChikatamarlaSS+KarlinIV-PhysRevLett} set about the problem of LB stencil construction from a discrete form
    of Boltzmann's H-theorem---in which the  $(-H)$  quantity  represents  a  sort  of  generalized  thermodynamic  entropy  for
    non-equilibrium states in the Boltzmann Gas Limit (BGL) that increases according to the second law of  thermodynamics  until
    equilibrium is reached~\cite{2011-HarrisS-Dover}. By maximizing the  entropy,  i.e.,  by  minimizing  $H$  in  the  discrete
    description,%
    %
    \begin{equation}
        H = \sum_{i=1}^{N}f_i\ln\left(\frac{f_i}{W_i}\right),
    \end{equation}
    %
    \noindent with appropriately chosen weights $W_i$ under a set of macroscopic constraints of mass and energy conservation  as
    well as constitutive relations for higher-order macroscopic tensors, authors arrived at \emph{explicit} expressions for  the
    weights $W_i$ and for the stencil reference temperature for several one-dimensional  lattice  models  having  from  3  to  5
    velocities. Methods obtained by this systematic were named `entropic' LB methods (ELBM).

    Remarkably, due to a pattern in Gauss-Hermite  quadrature~\cite{2003-AnsumaliS+OettingerHC-EuroPhysLett},  Chikatamarla  and
    Karlin       proposed       a       straightforward       way       of       obtaining       higher-dimension        lattice
    stencils~\cite{2006-ChikatamarlaSS+KarlinIV-PhysRevLett}:

    \begin{quote}
        \swshape
        ``[...]~discrete velocities [$\mathbf{c}_i$] in the $D$-dimensional case are  tensor  products  of  $D$  copies  of  the
        one-dimensional velocities, whereas the corresponding weights  [$W_i$]  are  algebraic  products  of  the  corresponding
        weights in one dimension.''
    \end{quote}

    As the immediate examples of~\cite{2006-PhilippiPC+SurmasR-PhysRevE} evidence, as soon as one allows for  including  several
    energy levels (but not necessarily all, nor in monotonic order) in a Bravais lattice,  velocity  count  no  longer  uniquely
    identifies lattices, and thus, \emph{velocity count-based lattice naming schemes are  bound  to  be  ambiguous  and  require
    additional information as to uniquely identify the lattice.\/} Usually, the additional information is laying out  \emph{all}
    velocity vectors, whether by energy level listings, a quiver-like lattice picture, or  tabulated  lattice  velocities  (plus
    weights, and either scaling factor or reference temperature)---all of which seem, to varying degrees, excessively wordy  and
    lengthy. Moreover, the very need for providing additional information as to uniquely identify an already named lattice seems
    to defeat the purpose of naming it, at least in part.


    %---------------------------------------------------------------------------------------------------------------------------
    \subsection{Higher-order lattice proliferation}


    \vspace{2.0mm}\noindent\textbf{Late 2000's:}\vspace{1.0mm}

    The onset of systematic techniques  for  LB  stencils  fabrication  in  2006~\cite{2006-ChikatamarlaSS+KarlinIV-PhysRevLett,
    2006-PhilippiPC+SurmasR-PhysRevE, 2006-ShanX+ChenH-JFluidMech}, allied to the expansion of LB  simulation  applications  and
    domains, contributed to the proliferation of LB stencils over the following years. This further highlighted the  ambiguities
    in velocity count-based naming schemes, as well as prompted the appearance of further lattice naming variety.

    The  following  `DdVb'  lattices   were   derived   by   Ortiz~\cite{2007-OrtizCEP-DrUFSC},   using   prescribed   abscissas
    quadrature~\cite{2006-PhilippiPC+SurmasR-PhysRevE}: second-order hexagonally regular (Bravais) D2V7; third-order  irregular,
    i.e., not space-filling, D2V12; fourth-order irregular D2V19, D2V20, D2V21, and regular D2V37; fifth-order irregular D2V28a,
    D2V28b, and regular D2V53a and D2V53b; sixth-order regular D2V81.  In  three-dimensional  Euclidean  space,  the  following:
    second-order irregular D3V13, third-order irregular D3V27, and fourth-order irregular D3V52 and D3V53.

    The D2V37, D2V53 and D2V81 lattices appear on~\cite{2007-PhilippiPC+DosSantosLOE-IntJModPhysC}. Analytically  derived  exact
    weights and scale factors for the D2V17 and D2V37 lattices are reported in~\cite{2007-SiebertDN+PhilippiPC-IntJModPhysC}.

    A `59-velocity model in three dimensions' is said to be of third order Hermite expansion, with sixth-order  tensor  isotropy
    in~\cite{2008-ChenH+ShanX-PhysD}; however, no weights, velocity  list  nor  scaling  factors  of  such  lattice  are  given.
    One-dimensional  D1Q3,  D1Q4,  D1Q5,  and  D1Q6,  as  well  as  two-dimensional  D2Q12  and  D2Q21  lattices  are  presented
    in~\cite{2008-KimSH+BoydID-JComputPhys},      with      finite      Knudsen      number      applications      in      view.
    Reference~\cite{2008-RubinsteinR+LuoLS-PhysRev} studies various two- and three-dimensional lattices of up to 51  velocities,
    while referring to a `DdQq' notation as being `standard'.

    On patent~\cite{2008-ShanX+ZhangR-USPat}, authors designate Cartesian,  space-filling  models  with  21,  37,  39,  and  103
    velocities, as `2D-1', `2D-2', `3D-1', and `3D-2', respectively.

    Discussion on a plurality of lattices and  lattice  operations,  such  as  stretching,  extending  (product),  pruning,  and
    superimposing, takes place in~\cite{2009-ChikatamarlaSS+KarlinIV-PhysRevE}; noteworthy ones are the  D$1$Q$(1+2k)$  lattices
    for $k \in \{1, \ldots, 4\}$, the higher-order ones also referred to `1D seven/nine velocity set'; the `ZOT', i.e.,  `\{$0$,
    $\pm 1$, $\pm 3$\}', or  `zero-one-three'  1-D  lattice;  the  `D1Q5-ZOT  lattice',  defined  as  ``{\swshape  the  shortest
    integer-valued        discrete        velocity        set        in        the        family        of         five-velocity
    sets\/}''~\cite{2006-ChikatamarlaSS+KarlinIV-PhysRevLett};                          the                          `D2Q25-ZOT'
    lattice~\cite{2008-ChikatamarlaSS+KarlinIV-CompPhysComm}, of eight-order isotropy,  obtained  by  extending  the  `D1Q5-ZOT'
    through a tensor product with itself, so that%
    %
    \begin{equation}
        D2Q25\mbox{-ZOT} \equiv (D1Q5\mbox{-ZOT}) \otimes (D1Q5\mbox{-ZOT});
    \end{equation}
    %
    \noindent the `D3Q125-ZOT lattice', also of eight-order iso\-tro\-py, obtained by extending either  the  `D1Q5-ZOT'  or  the
    `D2Q25-ZOT' lattice through a tensor product between them, so that%
    %
    \begin{equation}
        D3Q125\mbox{-ZOT} \equiv (D2Q25\mbox{-ZOT}) \otimes (D1Q5\mbox{-ZOT});
    \end{equation}
    %
    \noindent and also it's pruned version `D3Q41-ZOT', obtained by pruning, or, symmetrically removing velocity subsets.

    It is worth noting that pruning operations deal with discrete  velocity  groups  of  \emph{same  magnitude};  hence,  energy
    levels---so that pruning remove entire energy levels from a departure lattice configuration.

    Several lattice stencils are  given  in~\cite{2009-SurmasR+PhilippiPC-EurPhysJSpecialTopics}  \emph{mostly}  in  the  `DdVb'
    naming scheme from~\cite{2006-PhilippiPC+SurmasR-PhysRevE}, but also including an `n'  suffix,  as  in  D1V9n,  D2V28n,  and
    D3V53n, as to indicate the lattice is not space-filling, and also in the `\{$0$, $\pm a$, $\pm b$\}' format,  in  which  $b$
    can be an explicit multiple of $a$, as in \{$0$, $\pm a$, $\pm 3a$\}, naming schemes.  Bravais  lattices  are  given  up  to
    D1V15, D2V53, and D3V107 in one-, two-, and three- Euclidean spacial dimensions. Appendix tables list full velocity sets  as
    the `DdVb[n]' naming scheme, in which the `n' suffix is optional, is not uniquely determined.


    \vspace{2.0mm}\noindent\textbf{From 2010 to 2020:}\vspace{1.0mm}

    The      following      lattices      are      referenced      in       the       following       works:       a       D2Q36
    in~\cite[p.~452]{2010-AidunCK+ClausenJR-AnnuRevFluidMech}; a different D2V25,  D2V33,  D2V29-\{l,  r,  rl\},  D3V39,  D2V45,
    D2V77, rectangular D2Q13R, hexagonal D2V19H (also referred to as `GBL', after Grosfils, Boon and Lallemand), D2V55H, D2V85H,
    and D2V115H, the last three of fifth, sixth, and seventh order, respectively, and D3V107 in~\cite{2010-HegeleLA-DrUFSC};  an
    unnamed  one,  described  as   `a   ninth-order   accurate   Gauss-Hermite   quadrature   formula   in   three   dimensions'
    in~\cite{2008-NieX+ChenH-PhysRevE} having 121 velocities in three-dimensional Euclidean space.

    Spherical shell `SLB$(N; K, L, M)$' lattices---of order $N$,  $K$  spherical  shells,  $L$  shell  latitudes,  each  $K$-$L$
    intersection circle with $M$ uniformly distributed  discrete  velocities,  so  that  models  have  $K  \times  L  \times  M$
    velocities---with $1 \leqslant N \leqslant  7$,  $K,  L  >  N$,  and  $M  >  2N$,  i.e.,  SLB(1;~2,~2,~3),  SLB(2;~3,~3,~5),
    SLB(3;~4,~4,~7), SLB(4;~5,~5,~9), and so  on  up  to  SLB(7;  8,  8,  15),  and  even  an  SLB(N;  20,  20,  17)  are  given
    in~\cite{2012-AmbrusVE+SofoneaV-PhysRevE}, the last ones having, respectively, 960 and 6800 velocities!

    Rhombic     D2Q9,     rectangular     D2R11,     and     orthorhombic     Bravais     D3R23     lattices      are      found
    in~\cite{2013-HegeleJr+PhilippiPC-JSciComput}.

    Mattila  and  coauthors~\cite{2013-MattilaKK+PhilippiPC-IntJModPhysC}  have  shown  that  spurious  currents  emerge   along
    liquid-vapor interface in multiphase simulations. They have shown that  higher-order  stencils,  such  as  the  fourth-order
    D2V37, yields more localized and isotropic spurious currents than lower order ones, such as third-order D2V17 and  D2Q25-ZOT
    ones.    Thus,    multiphase    flows    became    another    application    requiring    multi-speed,    higher-order    LB
    methods~\cite{2014-SiebertDN+MattilaKK-PhysRevE}.       In        fact,        shortly        after,        the        group
    published~\cite{2014-MattilaKK+PhilippiPC-SciWorldJ} pre\-scribed ab\-scis\-sas-de\-rived D2V81 and D2V141  lattices,  along
    with an equivalent, however far simpler, form of the prescribed abscissas method in their Section~2.

    Reference~\cite{2014-MengJ+ZhangY-JComputPhys} lists lattice  velocities,  weights  and  scaling  factor  for  D2Q16,  D2Q17
    (reference~\cite{2007-SiebertDN+PhilippiPC-IntJModPhysC}'s D2V17 and reference~\cite{2010-ShanX-PhysRevE}'s $E_{2,7}^{17}$),
    D2Q37   (reference~\cite{2006-PhilippiPC+SurmasR-PhysRevE}'s   D2V37),   and   for   a   D3Q121,   originally   unnamed   on
    reference~\cite{2008-NieX+ChenH-PhysRevE}. Situations like this---in which a given lattice is referred to by different names
    in different sources yet without ruling out ambiguities---illustrate the need for improved lattice naming schemes.

    The `D$d$Q$b^d$' notation---as in D$3$Q$5^3$, D$2$Q$7^2$, D$3$Q$7^3$, and D$3$Q$11^3$---with  a  more  explicit  origin  and
    relationship with a lower-dimensionality, entropic `D$1$Q$b$' lattice of $b$ velocities, through  tensor  product  extension
    from         it~\cite{2006-ChikatamarlaSS+KarlinIV-PhysRevLett}---appears         in~\cite{2015-FrapolliN+KarlinIV-PhysRevE,
    2016-FrapolliN+KarlinIV-PhysRevE} in connection to compressible flow applications. In this scheme, if%
    %
    \begin{align}
        (D1Qb)^d & \equiv \;\underbrace{(D1Qb) \otimes (D1Qb) \ldots}_{\mbox{($d$ times)}}\;\mbox{, then} \\
        DdQb^d   & \equiv (D1Qb)^d.
    \end{align}
    %
    It is worth noting that the D$3$Q$7^3$ and D$3$Q$11^3$ lattices have 343 and 1331 velocities, respectively.

    Supersonic and hypersonic flow speeds are comparable to and greater than molecular thermal  velocity  scales,  respectively;
    moreover, many supersonic flows have a well defined prevailing flow direction, especially when simple, slender objects  move
    with high speeds through quiescent media. Changing from a rest to the object's  reference  frame  causes  molecule  velocity
    populations   to   be    shifted    by    the    object's    speed.    References~\cite{2016-FrapolliN+KarlinIV-PhysRevLett,
    2020-FrapolliN+KarlinIV-Entropy} present the $D2Q7^2$ lattice, i.e.,  a  D2Q49  one,  in  the  (i)~symmetric  variety,  rest
    reference frame, and (ii)~shifted variety, comoving reference frame  of  $U_x  =  1$  lattice  units.  Better  yet,  authors
    demonstrate that departing from a Galilean-invariant symmetric, rest  reference  frame  lattice,  reference-frame  shiftings
    \emph{do not change} the lattice weights, meaning%
    %
    \begin{equation}
        W_i(U, T) = W_i(0, T),
    \end{equation}
    %
    \noindent for arbitrary $U$, where $W_i$'s are the lattice weights in terms of the lattice reference speed and  temperature,
    and $U$ is the reference frame speed shifting. Better still, the Galilean invariance property allow for the construction  of
    higher-order lattices through tensor products of lower-order Galilean invariant ones, whether they are shifted or not.

    Therefore, let a symmetrical $D1Q7$ lattice, with velocity abscissas $V_7 = \{-3,$ $-2,$  $-1,$  $0,$  $+1,$  $+2,$  $+3\}$,
    produce a unit $U_x$-shifted lattice with velocity abscissas $V_7' = \{-2,$ $-1,$ $0,$ $+1,$ $+2,$ $+3,$  $+4\}$,  then  the
    velocity set of the unit $U_x$-shifted $D2Q7^2$ lattice is given by $V_{7x}' \otimes V_{7y}$.

    Body-centered cubic, BCC, lattice arrangements arising from emphasizing spatial discretization over the momentum one in  the
    discretization of the Boltzmann equation in~\cite{2016-NamburiM+AnsumaliS-SciRep} led to  BCC  lattices  named  `RD3Q27'.  A
    novel    BCC    lattice    model    named    RD3Q67    is     also     proposed     in~\cite{2018-AtifM+AnsumaliS-PhysRevE}.
    Reference~\cite{2017-LiL+KlausnerJF-IntJHeatMassTran} adds capital roman numerals to  `DdQb'  lattices  depending  on  their
    underlying relaxation time scheme in Multiple Relaxation Time, MRT, models, thus yielding D2Q9-I to  D2Q9-IV  lattice  model
    designations.

    Velocity count-based lattice naming ambiguities arose in~\cite{2017-PengY+ZhangJM-MathProblEng}, in which the different  yet
    homonymous            `D1Q5'            lattices            from            references~\cite{1998-QianYH+ZhouY-EuroPhysLett,
    2006-ChikatamarlaSS+KarlinIV-PhysRevLett}---one with $\{0,$ $\pm 1,$ $\pm 2\}$, and the other with $\{0,$ $\pm 1,$ $\pm 3\}$
    velocity sets---were distinguished by an `A' or `B' suffix, making them D1Q5A and D1Q5B.

    Finally, a space-filling regular Bravais D2V169 lattice, comprised of 169 velocity vectors, with corresponding  weights  and
    scaling factor appears in~\cite[p.~68]{2019-daRosaTG-MEngUTFPR}.


%------------------------------------------------------------------------------------------------------------------------------%
%                                                          Discussion                                                          %
%------------------------------------------------------------------------------------------------------------------------------%

\section{Discussion}

    From the lattice naming systems survey of the previous Section, one finds that many lattice  naming  schemes  have  appeared
    over the years, with each new variety either introduced as to accommodate or reflect a new aspect brought in  the  research,
    as  with~\cite{2006-ShanX+ChenH-JFluidMech,   2008-ChikatamarlaSS+KarlinIV-CompPhysComm,   2006-PhilippiPC+SurmasR-PhysRevE,
    2012-AmbrusVE+SofoneaV-PhysRevE, 2016-NamburiM+AnsumaliS-SciRep}, or to organize and distinguish multiple lattices within  a
    publication,        as        with~\cite{1991-QianYH+LallemandP-AdvKinTheoContMech,        2006-PhilippiPC+SurmasR-PhysRevE,
    2007-OrtizCEP-DrUFSC, 2014-MengJ+ZhangY-JComputPhys, 2017-LiL+KlausnerJF-IntJHeatMassTran}.

    The present survey is unaware of any published effort in the direction of major  standardizations  across  multiple  lattice
    types  and  features,  as  well  as  of  the  existence  of  any  concise  \emph{naming}\footnote{As  opposed  to   velocity
    \emph{listing}.} system that would allow for uniqueness by ruling out name ambiguity.

    Lattice naming variety appears to be due to (i)~the inherent decentralized nature of research, (ii)~the inherent novelty and
    discovery associated to the practice of research, making future features,  ideas,  and  demands  unforeseeable  to  previous
    studies, as well as to (iii)~the variety of lattice \emph{types}.

    On this last aspect, the history of the method has  seen  (a)~space-filling,  regular,  Bravais  types  in  linear,  square,
    triangular (hexagonal), cubic, and projected hypercubic geometries; (b)~irregular, non-space-filling ones;  (c)~those  based
    on spherical coordinate systems; (d)~those with shifted reference velocity frame; and  (e)~those  more  heavily  focused  on
    spatial space discretization rather than on momentum space. This facet alone may  at  best  difficult  efforts  in  creating
    concise, unambiguous lattice naming schemes of general scope.

    With respect to the continuation and adoption of lattice naming systems, the Euclidean dimension,  velocity  counting  based
    `DdQb'    template    of    1990    due    to    Qian~\cite{1990-QianYH-Paris},    and    of    1991     of     Qian     and
    coauthors~\cite{1991-QianYH+LallemandP-AdvKinTheoContMech} seems to be the closest thing to a present-time de-facto standard
    for   LB   stencil   naming,   being   thus   acknowledged   on   research~\cite{2008-RubinsteinR+LuoLS-PhysRev}   and    on
    review~\cite{2010-AidunCK+ClausenJR-AnnuRevFluidMech}   papers,   as   well   as    textbooks~\cite{2011-MohamadAA-Springer,
    2018-KrugerT+ViggenEM-Springer}. Nonetheless, reaching this current status hasn't been a quick process, as it seemingly took
    a considerable amount of years until the naming scheme became widely adopted in the LB literature, as evidenced by the  lack
    of its usage in the 1998 review paper of Chen and coauthors, which refer to some of them as `LBM models based on 21  and  25
    velocities'~\cite[p.~357]{1998-ChenS+DoolenGD-AnnuRevFluidMech}, and on the 2002 review paper by Succi and coauthors,  which
    refer to Qian and coworkers' model as `LBGK'~\cite[p.~1215]{2002-SucciS+ChenH-RevModPhys} after the collision model.

    Moreover,    other    naming    conventions    such    as    the     `$E_{D,n}^{d}$'     one     due     to     Shan     and
    coworkers~\cite{2006-ShanX+ChenH-JFluidMech},       the       `DdVb'       one       due       to        Philippi        and
    coworkers~\cite{2006-PhilippiPC+SurmasR-PhysRevE}, and the `DdQb'-based variations such as `-ZOT' suffix and  integer  power
    velocity count ones due to Chikatamarla and Karlin~\cite{2008-ChikatamarlaSS+KarlinIV-CompPhysComm} frequently re-appear  in
    many subsequent publications, but its adoption seem to be more or less confined to the proposing author's  research  groups,
    and to direct citations---so that one may perceive then to be in competition.

    It is worth noting that all `mainstream' lattice naming schemes---whether `$E_{D,n}^{d}$', `DdQb' and  its  variations---are
    able to describe space-filling, Bravais lattices. Yet, all  such  lattice  naming  schemes  are  velocity  count  based  and
    therefore suffer from ambiguity, as, for instance, the sole `D2Q25' (or `D2V25', or `D2Q$5^2$') information \emph{can}  mean
    many different lattice configurations, having completely different envelope shapes, conception strategy,  set  of  populated
    energy levels, and optional shiftings, since it only specifies a set of  25  discrete  velocities  in  two  Euclidean  space
    dimensions, and, in the case of the `$E_{D,n}^{d}$' scheme, the resulting order of approximation.

    Aiming at space-filling, regular, Bravais types in one- to three-dimensional Euclidean spaces with complete, fully-populated
    energy levels,  the  authors  conjecture~\cite{2020-NaaktgeborenC+AndradeFN-BravLatNam-engrXiv-rev00}  that  a  scheme  with
    (i)~energy-level-based primitives, (ii)~that allows for operations such as (tensor product)  extensions  and  shiftings;  is
    able to produce relatively \emph{concise} and \emph{unambiguous} lattice  names,  while  being  sufficiently  \emph{generic}
    within its category.


%------------------------------------------------------------------------------------------------------------------------------%
%                                                      Citations by Year                                                       %
%------------------------------------------------------------------------------------------------------------------------------%

\section{Citations by Year}

    The following are the citations indexed by year in chronological order: #----------------------------------------------------------------------#
#                                 1949                                 #
#----------------------------------------------------------------------#
1949-GradH-CommPureApplMath
#----------------------------------------------------------------------#
#                                 1954                                 #
#----------------------------------------------------------------------#
1954-BhatnagarPL+KrookM-PhysRev
#----------------------------------------------------------------------#
#                                 1973                                 #
#----------------------------------------------------------------------#
1973-HardyJ+PazzisO-JMathPhys
#----------------------------------------------------------------------#
#                                 1976                                 #
#----------------------------------------------------------------------#
1976-HardyJ+PomeauY-PhysRevA
#----------------------------------------------------------------------#
#                                 1986                                 #
#----------------------------------------------------------------------#
1986-FrischU+PomeauY-PhysRevLett
1986-WolframS-JStatPhys
#----------------------------------------------------------------------#
#                                 1987                                 #
#----------------------------------------------------------------------#
1987-FrischU+RivetJP-ComplexSyst
1987-SucciS-JPhysAMathGen
#----------------------------------------------------------------------#
#                                 1988                                 #
#----------------------------------------------------------------------#
1988-McNamaraGR+ZanettiG-PhysRevLett
#----------------------------------------------------------------------#
#                                 1989                                 #
#----------------------------------------------------------------------#
1989-HigueraFJ+JimenezJ-EuroPhysLett
1989-HigueraFJ+SucciS-EuroPhysLett
#----------------------------------------------------------------------#
#                                 1990                                 #
#----------------------------------------------------------------------#
1990-BenziR+VergassolaM-EuroPhysLett
1990-BenziR+VergassolaM-NuclPhysB
1990-CancelliereA+SucciS-PhysFluidsA
1990-QianYH-Paris
1990-VergassolaM+SucciS-EuroPhysLett
#----------------------------------------------------------------------#
#                                 1991                                 #
#----------------------------------------------------------------------#
1991-AppertC+ZaleskiS-PhysD
1991-BoonJP-PhysD
1991-ChenS+RoseH-PhysD
1991-CornubertR+LevermoreD-PhysD
1991-ErnstMH-PhysD
1991-FrischU-PhysD
1991-GunstensenAK+ZanettiG-PhysRevA
1991-KoelmanJMVA-EuroPhysLett
1991-QianYH+LallemandP-AdvKinTheoContMech
1991-SucciS+BenziR-PhysRevA
#----------------------------------------------------------------------#
#                                 1992                                 #
#----------------------------------------------------------------------#
1992-BenziR+VergassolaM-PhysRep
1992-QianYH+LallemandP-EuroPhysLett
#----------------------------------------------------------------------#
#                                 1993                                 #
#----------------------------------------------------------------------#
1993-AlexanderFJ+SterlingJD-PhysRevE
1993-QianYH-JSciComput
1993-QianYH+OrszagSA-EuroPhysLett
#----------------------------------------------------------------------#
#                                 1994                                 #
#----------------------------------------------------------------------#
1994-ChenY+AkiyamaM-PhysRevE
#----------------------------------------------------------------------#
#                                 1996                                 #
#----------------------------------------------------------------------#
1996-HeX+DemboM-JComputPhys
#----------------------------------------------------------------------#
#                                 1997                                 #
#----------------------------------------------------------------------#
1997-HeX+LuoLS-PhysRevE-APriori
#----------------------------------------------------------------------#
#                                 1998                                 #
#----------------------------------------------------------------------#
1998-ChenS+DoolenGD-AnnuRevFluidMech
1998-PavloP+VahalaL-PhysRevLett
1998-QianYH+ZhouY-EuroPhysLett
#----------------------------------------------------------------------#
#                                 2001                                 #
#----------------------------------------------------------------------#
2001-dHumieresD+LallemandP-PhysRevE
#----------------------------------------------------------------------#
#                                 2003                                 #
#----------------------------------------------------------------------#
2003-AnsumaliS+OettingerHC-EuroPhysLett
2003-LiboffRL-bookSpringer
2003-NourgalievRR+JosephD-IntJMulFlow
2003-WatariM+TsutaharaM-PhysRevE
#----------------------------------------------------------------------#
#                                 2005                                 #
#----------------------------------------------------------------------#
2005-LuXY-IntJModPhysC
#----------------------------------------------------------------------#
#                                 2006                                 #
#----------------------------------------------------------------------#
2006-ChikatamarlaSS+KarlinIV-PhysRevLett
2006-ChikatamarlaSS+KarlinIV-PhysRevLett-ELBM3D
2006-PhilippiPC+SurmasR-PhysRevE
2006-ShanX+ChenH-JFluidMech
2006-WatariM+TsutaharaM-PhysA
#----------------------------------------------------------------------#
#                                 2007                                 #
#----------------------------------------------------------------------#
2007-OrtizCEP-DrUFSC
2007-PhilippiPC+DosSantosLOE-IntJModPhysC
2007-SiebertDN+PhilippiPC-IntJModPhysC
2007-WatariM-PhysA
#----------------------------------------------------------------------#
#                                 2008                                 #
#----------------------------------------------------------------------#
2008-ChenH+ShanX-PhysD
2008-ChikatamarlaSS+KarlinIV-CompPhysComm
2008-KimSH+BoydID-JComputPhys
2008-NieX+ChenH-PhysRevE
2008-RubinsteinR+LuoLS-PhysRev
2008-ShanX+ZhangR-USPat
#----------------------------------------------------------------------#
#                                 2009                                 #
#----------------------------------------------------------------------#
2009-ChikatamarlaSS+KarlinIV-PhysRevE
2009-SurmasR+PhilippiPC-EurPhysJSpecialTopics
#----------------------------------------------------------------------#
#                                 2010                                 #
#----------------------------------------------------------------------#
2010-AidunCK+ClausenJR-AnnuRevFluidMech
2010-AidunCK+ClausenJR-AnnuRevFluidMech-supl
2010-HegeleLA-DrUFSC
2010-ShanX-PhysRevE
#----------------------------------------------------------------------#
#                                 2011                                 #
#----------------------------------------------------------------------#
2011-HarrisS-Dover
2011-MohamadAA-Springer
#----------------------------------------------------------------------#
#                                 2012                                 #
#----------------------------------------------------------------------#
2012-AmbrusVE+SofoneaV-PhysRevE
#----------------------------------------------------------------------#
#                                 2013                                 #
#----------------------------------------------------------------------#
2013-HegeleJr+PhilippiPC-JSciComput
2013-MattilaKK+PhilippiPC-IntJModPhysC
#----------------------------------------------------------------------#
#                                 2014                                 #
#----------------------------------------------------------------------#
2014-MattilaKK+PhilippiPC-SciWorldJ
2014-MengJ+ZhangY-JComputPhys
2014-SiebertDN+MattilaKK-PhysRevE
#----------------------------------------------------------------------#
#                                 2015                                 #
#----------------------------------------------------------------------#
2015-FrapolliN+KarlinIV-PhysRevE
#----------------------------------------------------------------------#
#                                 2016                                 #
#----------------------------------------------------------------------#
2016-FrapolliN+KarlinIV-PhysRevE
2016-FrapolliN+KarlinIV-PhysRevLett
2016-NamburiM+AnsumaliS-SciRep
2016-PhilippiPC+MattilaKK-JBrazSocMechSci
#----------------------------------------------------------------------#
#                                 2017                                 #
#----------------------------------------------------------------------#
2017-LiL+KlausnerJF-IntJHeatMassTran
2017-PengY+ZhangJM-MathProblEng
#----------------------------------------------------------------------#
#                                 2018                                 #
#----------------------------------------------------------------------#
2018-AtifM+AnsumaliS-PhysRevE
2018-KrugerT+ViggenEM-Springer
#----------------------------------------------------------------------#
#                                 2020                                 #
#----------------------------------------------------------------------#
2020-FrapolliN+KarlinIV-Entropy



%------------------------------------------------------------------------------------------------------------------------------%
%                                                         Conclusions                                                          %
%------------------------------------------------------------------------------------------------------------------------------%

\section{Conclusions}

    A survey of lattice naming systems for  lattice-Boltzmann  (LB)  methods,  from  its  Lattice-Gas  Automata  (LGA)  historic
    predecessor to the present time has been performed, which correspond to the period of years of our Lord from 1973 to 2020.

    From the survey, key findings include: (i)~the appearance (and discontinuance) of  many  lattice  naming  schemes  over  the
    years, (ii)~an apparent lack of published efforts solely geared  towards  major  lattice  name  standardizations,  (iii)~the
    existence of a great diversity of lattice types, (iv)~the prominence  of  Qian's  (and  coworkers)'~\cite{1990-QianYH-Paris,
    1991-QianYH+LallemandP-AdvKinTheoContMech} velocity-count based `DdQb' naming  scheme---such  as  D2Q9---being  the  closest
    thing to a de-facto standard in the LB literature; (v)~the existence of other, seemingly competing,  velocity  count  naming
    standards;   and   (vi)~the   ambiguity   of   velocity-count   based    lattice    naming    schemes,    plainly    evident
    in~\cite{2014-MengJ+ZhangY-JComputPhys, 2017-PengY+ZhangJM-MathProblEng}.

    From the survey and from the diversity of lattice types, it becomes somewhat  clear  that  (a)~probably  there  will  be  no
    generic and concise `one-size-fits-all' naming scheme for all surveyed models, let alone, published ones, and (b)~a concise,
    unambiguous naming scheme, at least for the more regular lattice types is in order, as to enable the necessary  distinctions
    between    models    of    same    dimensionality    and    velocity    count.     An     upcoming     work     from     the
    authors~\cite{2020-NaaktgeborenC+AndradeFN-BravLatNam-engrXiv-rev00} is to make a proposition.


%-------------------------------------------------------------------------------------------------------------------------------
